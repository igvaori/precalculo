\chapter{99}

\photon \photon \photon \photon \photon \photon 


	\LaTeX{} está formado por un gran conjunto de macros de TeX, escrito por Leslie Lamport en 1984, con la intención de facilitar el uso del lenguaje de composición tipográfica, 

	\LaTeX{}, creado por Donald Knuth. Es muy utilizado para la composición de artículos académicos, tesis y libros técnicos, dado 

\section{Uno.Uno}

	\LaTeX{} está formado por un gran conjunto de macros de TeX, escrito por Leslie Lamport en 1984, con la intención de facilitar el uso del lenguaje de composición tipográfica, 

\begin{destacado}
		\LaTeX{}, creado por Donald Knuth. Es muy utilizado para la composición de artículos académicos, tesis y libros técnicos, dado que la calidad .
\end{destacado}





		\LaTeX{}, creado por Donald Knuth. Es muy utilizado para la composición de n.





\section{segunda}

\subsection{dos.uno}
\subsection{dos.dos}



\begin{theorem}
jhckjhsc

kmsdbCKBSKL

MKDSbcksbklCS	
\end{theorem}

\begin{definition}
jhckjhsc

kmsdbCKBSKL

MKDSbcksbklCS	
\end{definition}



\newpage

\section{nueva3}


\begin{myalertblock}{myalertblok}
	creado por Donald Knuth. Es muy utilizado para la composición de artículos académicos, tesis y libros
\end{myalertblock}


\begin{myblock}{myblok}
	\LaTeX{}, creado por Donald Knuth. Es muy utilizado para la composición de artículos académicos, tesis y libros
\end{myblock}



\begin{myexampleblock}{myexampleblok}
	\LaTeX{}, creado por Donald Knuth. Es muy utilizado para la composición de artículos académicos, tesis y libros
\end{myexampleblock}



\subsection{Estilos de letras}
\textbf{textbf} \dots \textsf{textsf} \dots \textit{textit} \dots
\textsl{textsl} \dots \textsc{textsc} \dots \texttt{texttt} \dots

\subsection{Tamaños de letras}
Si queremos escribir con distintos tamaños:
\begin{table}[htbp]
  \centering
  \begin{tabular}{lll}
    \tiny{tiny} & \normalsize{normalsize} & \LARGE{LARGE} \\
    \scriptsize{scriptsize} & \large{large} & \huge{huge} \\
    \footnotesize{footnotesize} & \Large{Large} & \Huge{Huge}\\
    \small{small}
  \end{tabular}
  \caption{Tabla tamaños de letras}
\end{table}

\section{ecc}

Esto es simpático:
\[ \underbrace{1 + \overbrace{2 + 3}^{5} + 4}_{10} \]

Esto más complejo:
\begin{displaymath}
  \int e^{x+y}dx dy \,\, = \,\,
  \int e^x e^y dxdy \,\, = \,\,
  \underbrace{\int e^x dx \, \int e^y dy}_{\stackrel{\stackrel{\uparrow}{Variables \; son \;independientes}}{Podemos \;realizar\; la\; integracion\; para\; x\; e\; y}}
\end{displaymath}





\section{math}
******************. \LaTeX{}

\begin{align}
x_1 &= 1 \\
x_2 &= -1 \\
x_3 &= \sqrt{2} \\
lalala  \notag \\
sin etiq.  \notag \\
con *  \tag{*} \\
mas  \tag{**}
\end{align}

*********** negaciones inclusiones

$$\mathbb{Z} \not\subset \mathbb{N},\ \mathbb{Z} \not\subseteq \mathbb{N}.
\qquad 
\mathbb{N} \not\supset \mathbb{Z},\ \mathbb{N} \not\supseteq \mathbb{Z}. $$

*************** def. funcion

$$\begin{array}{rccl}
f: & A & \longrightarrow & B \\
& x & \mapsto & x^2
\end{array}$$



******************. bmatrix y pmatrix

$$\begin{bmatrix}
x_{1} \\
x_{2} \\
\vdots \\
x_{n}
\end{bmatrix}
=
\begin{pmatrix}
x_1, x_2, \dots, x_n
\end{pmatrix}^{t}$$

*****************. norma vector: $\qquad \|\vec{v}\|=\sqrt{v_1^2+v_2^2+\ldots+v_n^2} \quad |\vec v|$


TACHAR número horizontalmente: $\quad \text{\textst{157}}$

\section{Chorradas}
\vspace{10mm}
\begin{center}
\fbox{\fbox{\parbox{10cm}{\centering
{\bf 25012 Teoría de Números}\\
{\sf Convocatoria: Enero 2017. Fecha: 11-01-2017}\\
Cuarto de grado en Matemáticas}}}
\end{center}

SANGRADO
\begin{quotation}
\begin{textit} % en cursiva
Cubum autem in duos cubos, aut quadratoquadratum in duos quadratoquadratos, et
generaliter nullam in infinitum ultra quadratum potestatem in duos eiusdem nominis fas
est dividere cuius rei demonstrationem mirabilem sane detexi, hanc marginis exiguitas non
caperet.
\end{textit}
\end{quotation}


TEXTO ROTADO $180^o$

\rotatebox{180}{\leftline{\textcolor{gris}{solo cabe una línea, no más}}}



RESUMEN

\begin{resumen}
	dslfhslajdhfkjlsdah f sjahf shf sohaf sdha fsadhf s klavjdkas.  iosp fois apfo shaof dsiopa foipds fiods opifiodsu fiods. uodsi fiods f dsio uiops oiudsaio  uo sdiof sdaoip posa opia oipaoipu opa iou iou usaio oipa oipa oa op

klavjdkas.  iosp fois apfo shaof dsiopa foipds fiods opifiodsu fiods. uodsi fiods f dsio uiops oiudsaio  uo sdiof sdaoip posa opia oipaoipu opa iou iou usaio oipa oipa oa op
\end{resumen}




CITA

\begin{cita}{Autor Lore Ipsum}
	klavjdkas.  iosp fois apfo shaof dsiopa foipds fiods opifiodsu fiods. uodsi fiods f dsio uiops oiudsaio  uo sdiof sdaoip posa opia oipaoipu opa iou iou usaio oipa oipa oa op klavjdkas.  iosp fois apfo shaof dsiopa foipds fiods opifiodsu fiods. uodsi fiods f dsio uiops oiudsaio  uo sdiof sdaoip posa opia oipaoipu opa iou iou usaio oipa oipa oa op
\end{cita}


\vspace*{\fill}
\begin{center}
  \fbox{Realizado en \LaTeX{}}
   \begin{figure}[H]
	\centering
	\includegraphics[width=.3
	\textwidth]{imagenes/firma.png}
\end{figure}
\today{}
\end{center}
\vspace{3cm}
